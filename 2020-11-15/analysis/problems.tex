\documentclass{article}

\usepackage{mathrsfs,amssymb,amsmath}

\begin{document}

\section{Show that wk is the smallest topology on $\mathscr{X}$ such that each $x^*$ in $\mathscr{X}^*$}

Must show that an arbitrary open set in wk can be generated by some collection of sets of the form $x^{*-1}(V)$.

Start with an arbitrary open set $U$. Since $X$ with wk is a locally convex set, $\bigcap^n_{i=1}\{x\epsilon \mathscr{X} : p_j( x-x_0) < \varepsilon_j\} \subseteq U$ for some finite list of $\varepsilon$ and $p$, where $p_{x^*}=|\langle x,x^*\rangle|$.

 $\bigcap^n_{i=1}\{x\in \mathscr{X} : p_j( x-x_0) < \varepsilon_j\}$ is open and the intersection of pre-images of open sets from $\mathbb{F}$ (why?).

Since $U$ is generated by the subbase of preimages of the collection of $x^*$ and $U$ is arbitrary, all open sets of wk are generated in this way, thus is the smallest possible topology.

\section{Show that wk* is the smallest topology on $\mathscr{X}^*$ such that for each $x$ in $\mathscr{X}$, $x^* \mapsto \langle x, x^* \rangle$}


This purported topology $\sigma$ is generated by open sets that are preimages of open sets of functions of the form $x^* \mapsto \langle x, x^* \rangle$. By continuity of composition, they are also the preimages of open sets of functions of the form $x^* \mapsto | \langle x, x^* \rangle | $. Open sets of $\mathbb{R}$ are generated by open balls, so the pre-images are generated by sets of the form $| \langle x, x^* \rangle | < \varepsilon $.

A set $U$ of $\mathscr{X}^*$ is weakly open if and only if for every $x^*_0$ in $U$ there is an $\varepsilon$ and there are $x_1, ... x_n$ in $\mathscr{X}$ such that 

$$\bigcap^n_{i=1}\{x^*\in \mathscr{X}^* : |\langle x_k, x^* - x^*_0 \rangle| < \varepsilon_j\} \subseteq U$$

Every such set $U$ can be generated as part of $\sigma$ and therefore is the smallest topology 

\section{Prove Theorem 1.3}

Theorem 1.3 is proven in the text!

\section{Let $\mathscr{X}$ be a complex LCS and let $\mathscr{X}^*_{\mathbb{R}}$ denote the collection of all continous real linear functionals on $\mathscr{X}$. Use the elements of $\mathscr{X}^*_{\mathbb{R}}$  to define seminorms on $\mathscr{X}$ and let $\sigma(\mathscr{X}, \mathscr{X}^*_{\mathbb{R}})$ be the corresponding topology. Show that $\sigma(\mathscr{X}, \mathscr{X}^*) = \sigma(\mathscr{X}, \mathscr{X}^*_{\mathbb{R}})$}

Take the function $f$

\begin{align*}
f: \mathbb{C} \mapsto \mathbb{R} \\
f(x) = |x|
\end{align*}

$f$ defines a function $\hat{f}$ from $\mathscr{X}^*$ to $\mathscr{X}^*_{\mathbb{R}}$ via composition. We will show that $\hat{f}$ is a homeomorphism.

$\hat{f}$ is clearly continuous because it is the composition of two continuous functions.

Open sets around $x^*_0$ in $\mathscr{X}^*$ are generated by sets of the form $\bigcap^n_{i=1}\{x^*\in \mathscr{X}^* : |\langle x_k, x^* - x^*_0 \rangle| < \varepsilon_j\}$ for some finite set of $x_k$ and $\varepsilon_j$. The preimage of this set under $\hat{f}^{-1}$ is ???, which is open, therefore $\hat{f}^{-1}$ is continuous.

\section{If $A \subseteq  \mathscr{X}$...}

\subsection{$A^{\circ}$ is convex and balanced}

$A^{\circ} \equiv \{x^*\in \mathscr{X}^*: | \langle a, x^*\rangle | \le 1 \text{ for all } a \text{ in } A\}$

To show that $A^{\circ}$ is balanced, given $x^*$ in $A^{\circ}$ and $|\alpha| \le 1$, we have to show that $\alpha x^*$ is also in $A^{\circ}$.

If $|\langle a, x^*\rangle | \le 1$, then $|\langle a, \alpha x^*\rangle | = |\alpha |  | \langle a, x^* \rangle | \le 1$, hence $\alpha x^*$ is in $A^{\circ}$

To show that $A^{\circ}$ is convex, we have to show that given arbitrary $x_1^*$ and $x_2^*$, then all of $\{t x_1^* + (1-t) x_2^* :0\le t \le 1\}$ is in $A^{\circ}$.

If $|\langle a, x_1^*\rangle | \le 1$ and $|\langle a, x_2^*\rangle | \le 1$, then $t|\langle a, x_1^*\rangle | \le t$ and $(1-t)|\langle a, x_2^*\rangle | \le 1-t$ as long as $0 \le t \le 1$ (to not flip the signs). Adding the inequalities we get $t|\langle a, x_1^*\rangle | +(1-t)|\langle a, x_2^*\rangle | \le 1$.

\section{If $A\subseteq\mathscr{X}$, show that $A$ is weakly bounded if and only if $A^{\circ}$ is absorbing in $\mathscr{X}^*$}

Assume $A$ is weakly bounded. 

Assume $A^{\circ}$ is absorbing in $\mathscr{X}^*$

\end{document}