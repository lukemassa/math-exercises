\documentclass{article}

\usepackage{mathrsfs,amssymb,amsmath}

\begin{document}

\section{If $\mathscr{A}$ is a Banach algebra with identity and $a \in \mathscr{A}$ and is nilpotent (that is, $a^n=0$ for some $n$), then $\sigma(a) = \{0\}$}

Firstly, assume $a$ is invertible. Then 

\begin{align*}
    a^n &= 0 \\
    (a^{-1})^n a^n &= (a^{-1})^n 0 \\
    1 &= 0
\end{align*}

a contradiction, so $0 \in \sigma(a)$.

Next, we need to show that $a-\alpha$ for any $\alpha \in \mathbb{C}$ is invertible.

If $1 + a + a^2 ...$ converges, it will converge to $1 / (1-a)$ by the geometric series. Since $a^n=0$, all terms $\ge n$ vanish, hence the series converges, and $1/(1-a) \in \mathscr{A}$

Since $\mathscr{A}$ is closed under multiplication, it also contains $1/(a-1)$, hence $a-1$ is invertible, so $1 \not \in \sigma(a)$. 

TODO: Extend to the other $\alpha \in \mathbb{C} \setminus 0 $ (maybe some simple algebra trick?)


\end{document}
