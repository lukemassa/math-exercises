\documentclass{article}

\usepackage{mathrsfs,amssymb,amsmath}

\begin{document}

\section{If $S:l_p \to l_p$ is defined by $S(\alpha_1, \alpha_2, ...) =(0, \alpha_1, \alpha_2, ...)$ describe the lattice of invariant closed subspaces of S}

$\mathscr{M}_n=\{x\in l^p:x(k)=0 \text{ for } 1 \le k \le n\}$, then $\mathscr{M}_n \in \text{Lat}S$, if $x \in \mathscr{M}_n$, so is $S(x)$, since they both start with at least $n$ 0s.

$\mathscr{M}_{n+1} \in \mathscr{M}_n$ since if a series begins w n+1 0s it will also begin with n zeros.

Claim: These subspaces, together with the zero element and all of $l^p$, represent all of Lat T (which is thus a totally ordered set).

Start with the element $x_1=(1, 0, 0, ...)$. It will be shown that the smallest closed invariant subspace $X$ that contains $x_1$ is fact the whole space.

If $x_1 \in X$, so is any element of the form $(\alpha_1, 0, 0, ...)$, since $X$ is a subspace so should be closed under scalar multiplication. Also, if $x_1 \in X$, so is $S(x_1) = (0, 1, 0, 0, ...) = x_2$, as well as $(0, \alpha, 0, 0)$. Since a subspace is closed under vector addition, all elements of the form $(\alpha_1, \alpha_2, \alpha_3, ..., 0, 0, ...)$ are thus in $X$. Call the set of all such points $Y \subset X$.

Now we must show that $Y$ is dense in $l_p$, hence that $X=l_p$. Given an arbitrary element of $l_p$ and $\epsilon$, since $l_p -> 0$, we can produce an element $y \in Y$ such that $|| l_p - y || < \epsilon$

By the same argument, any element whose first n entries are 0 will be identical to $\mathscr{M}_n$, thus proving the claim.


\end{document}
