\documentclass{article}

\usepackage{mathrsfs,amssymb,amsmath}

\begin{document}

\section{Why does Theorem VIII.1.14 not contradict Example VII.5.1}

Let $\mathscr{A} \subseteq \mathscr{B}$ be Banach Algebras.

Take $X \equiv \{a-\lambda I : \lambda \in \mathbb{C} \}$. By VII.5.4 (pg 207), $\sigma_{\mathscr{B}}(a) \subseteq \sigma_{\mathscr{A}}(a)$.

This can be interpreted as saying in general that there are elements of X that are not invertible in $\mathscr{A}$ but are invertible in $\mathscr{B}$ since there simply more elements in $\mathscr{B}$, that might be the inverse of an element in X.

However, if both $\mathscr{A}$ and $\mathscr{B}$ are $C^*$, then by the proof of VIII.1.14, going from $\mathscr{A}$ to $\mathscr{B}$ does not add any inverses.

Subsequently the elements of X that are invertible is the same when taken as a subset of $\mathscr{A}$ or $\mathscr{B}$, thus $\sigma_{\mathscr{B}}(a) = \sigma_{\mathscr{A}}(a)$.

\end{document}
