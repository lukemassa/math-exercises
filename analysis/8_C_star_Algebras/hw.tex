\documentclass{article}

\usepackage{mathrsfs,amssymb,amsmath}

\begin{document}

\section{Why does Theorem VIII.1.14 not contradict Example VII.5.1}

Take $\mathbb{T}$ as the unit circle in the complex plane, and take $\mathscr{A} = C(\mathbb{T})$ and $\mathscr{B}\subset \mathscr{A}$ the closure in $\mathscr{A}$ of polynomials in $z$. $\sigma_{\mathscr{A}}(z)=\mathbb{T}$ and $\sigma_{\mathscr{B}}(z)= cl \mathbb{T} =$ unit disk.

This appears to contradict VIII.1.14 since $\mathbb{B}$ is a subset of $\mathscr{A}$ with a common identity (namely the function that maps $\mathbb{T}$ to 1) and common norm (namely the sup norm), yet the spectra of a shared point (in this case z) differ.

However, the premise of VIII.1.14  requires both $\mathscr{A}$ and $\mathscr{B}$ to be C* algebras, and while $\mathscr{A}$ is closed under taking the adjoint, $\mathscr{B}$ is not.

Take z*, which as a function equals $z^{-1}/||z||$. The problem thus reduces to on what subset of the complex plane the function $1/z$ is holomorphic.

Here my complex analysis is too weak, but certainly the whole unit disk would be no good since it contains a pole. I believe the circle is also out because of some fact about winding numbers that I can't remember.

\end{document}
