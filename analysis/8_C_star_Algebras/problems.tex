\documentclass{article}

\usepackage{mathrsfs,amssymb,amsmath,amsfonts}

\begin{document}

\section{If $\mathscr{H}$ is a Hilbert space, $\mathscr{A}=\mathscr{B}(\mathscr{H} )$ is a C*-algebra where for each A in $\mathscr{B}(\mathscr{H} )$, A*=the adjoint of A}

VII.X.Y pg Z (TODO)

$\langle Ah, k \rangle = \langle h, A^*k \rangle = \overline{ \langle A^*k,h \rangle}$

$\langle A^*k, h \rangle = \langle k, A^{**}h \rangle = \overline{\langle A^{**}h,k \rangle}$

The above are justified by the definition of the adjoint and inner product, for all values h and k in $\mathscr{H}$.

Taking the complex conjugate of the bottom row shows that $\langle Ah, k \rangle = \langle A^{**}h,k \rangle$, hence A=A**.

$\langle (AB)^*h, k \rangle = \langle h, (AB)k \rangle = \langle h, A(B(k)) \rangle = \langle A^* h, B k \rangle = \langle B^*A^*h,k \rangle$

Hence, $(AB)^* = B^*A^*$

$\langle (\alpha A + B)^*h,k \rangle = \langle h, (\alpha A + B) k \rangle = \langle h, \alpha A k + B k \rangle$

$\langle \bar{\alpha}A^*h+B^*h, k \rangle$


\section{If X is a compact Hausdorff space, show that X is totally disconnected if and only if C(X) is the closed linear span of its projections ($\equiv$ hermitian idempotents)}

VII.X pg Y (TODO)

Assume $X$ is totally disconnected. To make things simpler assume $X=[0,1] \cap \mathbb{Q}$. Consider a bounded function $f$. Ignoring net vs sequence issues (TODO: is this justified?), we must show that $f$ can be written as the convergence of $a_1 e_1 + a_2 e_2 + ... $ where $a_n \in \mathbb{C}$ and $e_n$ are hermitian idempotents.

In this context, a hermitian idempotent is one whose range is $\{0,1\}$, since those are the only real numbers equal to their own square. Consider a series of function $e_{q_1}, e_{q_2}$, where for each ${q_i} \in X$, define

\begin{equation*}
e_{q_i}(x)= \left\{
        \begin{array}{ll}
            1 & \quad x = q_i \\
            0 & \quad x \ne q_i
        \end{array}
    \right.
\end{equation*}

$f$ can therefore be written as a countably infinite linear sum of such a series, thus is in the closed linear span of the projections.

Assume C(X) is the closed linear span of its projections. ???

\end{document}
