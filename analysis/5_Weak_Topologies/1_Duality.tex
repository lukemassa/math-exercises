\documentclass{article}

\usepackage{mathrsfs,amssymb,amsmath}

\begin{document}

Page whatever
\section{Show that wk is the smallest topology on $\mathscr{X}$ such that each $x^*$ in $\mathscr{X}^*$}

Must show that an arbitrary open set in wk can be generated by some collection of sets of the form $x^{*-1}(V)$.

Start with an arbitrary open set $U$. Since $X$ with wk is a locally convex set, $\bigcap^n_{i=1}\{x\epsilon \mathscr{X} : p_j( x-x_0) < \varepsilon_j\} \subseteq U$ for some finite list of $\varepsilon$ and $p$, where $p_{x^*}=|\langle x,x^*\rangle|$.

 $\bigcap^n_{i=1}\{x\in \mathscr{X} : p_j( x-x_0) < \varepsilon_j\}$ is open and the intersection of pre-images of open sets from $\mathbb{F}$ (why?).

Since $U$ is generated by the subbase of preimages of the collection of $x^*$ and $U$ is arbitrary, all open sets of wk are generated in this way, thus is the smallest possible topology.

\section{Show that wk* is the smallest topology on $\mathscr{X}^*$ such that for each $x$ in $\mathscr{X}$, $x^* \mapsto \langle x, x^* \rangle$}


This purported topology $\sigma$ is generated by open sets that are preimages of open sets of functions of the form $x^* \mapsto \langle x, x^* \rangle$. By continuity of composition, they are also the preimages of open sets of functions of the form $x^* \mapsto | \langle x, x^* \rangle | $. Open sets of $\mathbb{R}$ are generated by open balls, so the pre-images are generated by sets of the form $| \langle x, x^* \rangle | < \varepsilon $.

A set $U$ of $\mathscr{X}^*$ is weakly open if and only if for every $x^*_0$ in $U$ there is an $\varepsilon$ and there are $x_1, ... x_n$ in $\mathscr{X}$ such that 

$$\bigcap^n_{i=1}\{x^*\in \mathscr{X}^* : |\langle x_k, x^* - x^*_0 \rangle| < \varepsilon_j\} \subseteq U$$

Every such set $U$ can be generated as part of $\sigma$ and therefore is the smallest topology 

\section{Prove Theorem 1.3}

Theorem 1.3 is proven in the text!

\section{Let $\mathscr{X}$ be a complex LCS and let $\mathscr{X}^*_{\mathbb{R}}$ denote the collection of all continous real linear functionals on $\mathscr{X}$. Use the elements of $\mathscr{X}^*_{\mathbb{R}}$  to define seminorms on $\mathscr{X}$ and let $\sigma(\mathscr{X}, \mathscr{X}^*_{\mathbb{R}})$ be the corresponding topology. Show that $\sigma(\mathscr{X}, \mathscr{X}^*) = \sigma(\mathscr{X}, \mathscr{X}^*_{\mathbb{R}})$}

Goal is to show that an open set in $\sigma(\mathscr{X}, \mathscr{X}^*)$ is open in $\sigma(\mathscr{X}, \mathscr{X}^*_{\mathbb{R}})$ and vice versa.

Since $\mathscr{X}^*_{\mathbb{R}}$ is a strict subset of $\mathscr{X}^*$, $\sigma(\mathscr{X}, \mathscr{X}^*_{\mathbb{R}}) \subset \sigma(\mathscr{X}, \mathscr{X}^*)$ (since $\sigma(\mathscr{X}, \mathscr{X}^*)$ are those functions that make all linear functionals continuous, including the real linear functionals.)

Thus we need to show that $\sigma(\mathscr{X}, \mathscr{X}^*) \subset \sigma(\mathscr{X}, \mathscr{X}^*_{\mathbb{R}})$

Take the functions $f_r, f_i : \mathbb{C} \mapsto \mathbb{R}$ that map the complex numbers to their real, imaginary parts, respectively.

Open sets around $x_0$ in $\sigma(\mathscr{X}, \mathscr{X}^*)$ are generated by sets of the form $\bigcap^n_{i=1}\{x\in \mathscr{X} : |\langle x - x_0, x^*_k \rangle| < \varepsilon_k\}$ for some finite set of $x^*_k$ and $\varepsilon_k$. To simplify consider open sets generated by a single complex semi-norm $x^*_0$. We will show that these open sets are also generated by two real semi-norms, $x^*_0$ composed with $f_r$ and $f_i$. 

Firstly, observe that the open sets on the complex plane as a 1-dimensional complex vector space are the same as the open sets in $\mathscr{R}^2$ as a 2-dimensional real vector space since the homeomorphism between them simply ignores the complex structure (is this sufficient?). Since there's a unique combination of real part and imaginary part for each complex function, the images of the corresponding functions will be the same, hence the pre-images will be the same.

The same will be true given the finite intersection of sets generated by a finite number of seminorms. Thus any open set in $\sigma(\mathscr{X}, \mathscr{X}^*)$ is also open in $\sigma(\mathscr{X}, \mathscr{X}^*_{\mathbb{R}})$


\section{If $A \subseteq  \mathscr{X}$...}

\subsection{$A^{\circ}$ is convex and balanced}

$A^{\circ} \equiv \{x^*\in \mathscr{X}^*: | \langle a, x^*\rangle | \le 1 \text{ for all } a \text{ in } A\}$

To show that $A^{\circ}$ is balanced, given $x^*$ in $A^{\circ}$ and $|\alpha| \le 1$, we have to show that $\alpha x^*$ is also in $A^{\circ}$.

If $|\langle a, x^*\rangle | \le 1$, then $|\langle a, \alpha x^*\rangle | = |\alpha |  | \langle a, x^* \rangle | \le 1$, hence $\alpha x^*$ is in $A^{\circ}$

To show that $A^{\circ}$ is convex, we have to show that given arbitrary $x_1^*$ and $x_2^*$ in $A^{\circ}$, then an arbitrarily chosen $x_3$ in $\{t x_1^* + (1-t) x_2^* :0\le t \le 1\}$ is also in $A^{\circ}$, say $t_3 x_1^* + (1-t_3) x_2^*$ (where $t_3$ was chosen arbitrarily in $0\le t \le 1\}$).

Hence we need to prove that, for all $a$ in $A$, $|\langle a, t_3 x_1^* + (1-t_3) x_2^* \rangle | \le 1$

If $|\langle a, x_1^*\rangle | \le 1$ and $|\langle a, x_2^*\rangle | \le 1$, then simply by multiplication of both sides $t_3|\langle a, x_1^*\rangle | \le t_3$ and $(1-t_3)|\langle a, x_2^*\rangle | \le 1-t_3$ since $0 \le t_3 \le 1$ (to not flip the signs). Adding the inequalities we get $t_3|\langle a, x_1^*\rangle | +(1-t_3)|\langle a, x_2^*\rangle | \le 1$.

SOMEHOW: prove  $|\langle a, t_3 x_1^* + (1-t_3) x_2^* \rangle | \le t_3|\langle a, x_1^*\rangle | +(1-t_3)|\langle a, x_2^*\rangle |$ (Cauchy inequality + convexity of the reals?)

\subsection{If $A_1 \subseteq A$, then $A^{\circ} \subseteq A^{\circ}_1$}

If $x^* \in A^{\circ}$, then $| \langle a,x^* \rangle | \le 1$ for all $a$ in $A$. Since $A_1 \subseteq A$,  $| \langle a,x^* \rangle | \le 1$ for all $a$ in $A_1$ as well. Therefore $x^* \in A^{\circ}$. Since $x^*$ was chosen arbitrarily,  $A^{\circ} \subseteq A^{\circ}_1$.


\section{If $A\subseteq\mathscr{X}$, show that $A$ is weakly bounded if and only if $A^{\circ}$ is absorbing in $\mathscr{X}^*$}

Assume $A$ is weakly bounded. 

For every open set $U$ containing $0$, there is an $\varepsilon > 0$ such that $\varepsilon A \subseteq U$ (see definition 2.5 in IV. $\S$2, pg 106). Take one such open set $U_0$ and its associated $\varepsilon_0$. There exists $x^*_1, ..., x^*_n$ and $\hat{\varepsilon}_0$ such that 

$$\bigcap^n_{i=1}\{x \in \mathscr{X} : |\langle x, x^*_k \rangle| < \hat{\varepsilon} \} \subseteq U_0$$


hmmm, still stuck here.



For an arbitrary $x^*$, we have to show that $| \langle a, t x^* \rangle | \le 1$ for all $a$ in $A$ and $0 \le t < \varepsilon$.


\end{document}