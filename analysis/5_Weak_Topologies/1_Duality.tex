\documentclass{article}

\usepackage{mathrsfs,amssymb,amsmath}

\begin{document}

Page whatever
\section{Show that wk is the smallest topology on $\mathscr{X}$ such that each $x^*$ in $\mathscr{X}^*$}

Must show that an arbitrary open set in wk can be generated by some collection of sets of the form $x^{*-1}(V)$.

Start with an arbitrary open set $U$. Since $X$ with wk is a locally convex set, $\bigcap^n_{i=1}\{x\in \mathscr{X} : p_j( x-x_0) < \varepsilon_j\} \subseteq U$ for some finite list of $\varepsilon$ and $p$, where $p_{x^*}=|\langle x,x^*\rangle|$.

 $\bigcap^n_{i=1}\{x\in \mathscr{X} : p_j( x-x_0) < \varepsilon_j\}$ is open and the intersection of pre-images of open sets from $\mathbb{F}$, which are simply open balls of radius $\varepsilon_j$ around $x_0$. 

Since $U$ is generated by the subbase of preimages of the collection of $x^*$ and $U$ is arbitrary, all open sets of wk are generated in this way, thus is the smallest possible topology.

\section{Show that wk* is the smallest topology on $\mathscr{X}^*$ such that for each $x$ in $\mathscr{X}$, $x^* \mapsto \langle x, x^* \rangle$}


This purported topology $\sigma$ is generated by open sets that are preimages of open sets of functions of the form $x^* \mapsto \langle x, x^* \rangle$. By continuity of composition, they are also the preimages of open sets of functions of the form $x^* \mapsto | \langle x, x^* \rangle | $. Open sets of $\mathbb{R}$ are generated by open balls, so the pre-images are generated by sets of the form $| \langle x, x^* \rangle | < \varepsilon $.

A set $U$ of $\mathscr{X}^*$ is weakly open if and only if for every $x^*_0$ in $U$ there is an $\varepsilon$ and there are $x_1, ... x_n$ in $\mathscr{X}$ such that 

$$\bigcap^n_{i=1}\{x^*\in \mathscr{X}^* : |\langle x_k, x^* - x^*_0 \rangle| < \varepsilon_j\} \subseteq U$$

Every such set $U$ can be generated as part of $\sigma$ and therefore is the smallest topology 

\section{Prove Theorem 1.3}

Theorem 1.3 is proven in the text!

\section{Let $\mathscr{X}$ be a complex LCS and let $\mathscr{X}^*_{\mathbb{R}}$ denote the collection of all continous real linear functionals on $\mathscr{X}$. Use the elements of $\mathscr{X}^*_{\mathbb{R}}$  to define seminorms on $\mathscr{X}$ and let $\sigma(\mathscr{X}, \mathscr{X}^*_{\mathbb{R}})$ be the corresponding topology. Show that $\sigma(\mathscr{X}, \mathscr{X}^*) = \sigma(\mathscr{X}, \mathscr{X}^*_{\mathbb{R}})$}

Since every complex linear function is also real linear, $\mathscr{X}^*$ is a subset of $\mathscr{X}^*_{\mathbb{R}}$,  hence  $\sigma(\mathscr{X}, \mathscr{X}^*) \subset \sigma(\mathscr{X}, \mathscr{X}^*_{\mathbb{R}})$ 

Thus we need to show that $\sigma(\mathscr{X}, \mathscr{X}^*_{\mathbb{R}}) \subset \sigma(\mathscr{X}, \mathscr{X}^*)$.

An arbitrary open sets around $x_0$ in $\sigma(\mathscr{X}, \mathscr{X}^*_{\mathbb{R}})$ is generated by sets that are pre-images of open sets of a finite set of real-linear functions. Take one such real-linear function $f$ which takes an open neighborhood $U_0$ to some open neighborhood $V_0$, and consider some open rectangle. If we can show that $U_0$ is open in $\sigma(\mathscr{X}, \mathscr{X}^*)$, this will allow us to conclude that any arbitrary in $\sigma(\mathscr{X}, \mathscr{X}^*_{\mathbb{R}})$ is open in $\sigma(\mathscr{X}, \mathscr{X}^*)$.

If $f$ is real linear, then define $\hat{f}(z) \equiv f(z) - i f(iz)$. The image of $U_0$ under $\hat{f}$ is $U_0$ plus $ (-i) W_0$, where $W_0$ is the image of $U_0$ times $i$. Since an open set plus an arbitrary set in a $LCS$ is open, it doesn't matter what exactly $W_0$, the image will be open in $\mathbb{C}$ (is this sufficient?). Hence if $\hat{f}$ is shown to be complex linear, then $U_0$ is open in $\sigma(\mathscr{X}, \mathscr{X}^*)$ and the proof is concluded. 

\begin{align}
\hat{f}( \alpha z) &= f(\alpha z) - i f (\alpha i z) \\
&= f( (a+bi) z) - i f((a+bi) i z) \\
&= f(az + biz) - i f(aiz - bz) \\
&= f(az) + f(biz) - if(aiz) + if(bz) \\
&= af(z) + bf(iz) - iaf(iz) + bif(z) \\
&= (a+bi) (f(z) - i f(iz)) \\
&= \alpha (f(z) - i f(iz)) \\
&= \alpha \hat{f}(z)
\end{align}

5) because $f$ is real linear and $a$ and $b$ are real. 6) is just collecting terms. This shows that $\hat{f}$ is complex linear.

\section{If $A \subseteq  \mathscr{X}$...}

\subsection{$A^{\circ}$ is convex and balanced}

$A^{\circ} \equiv \{x^*\in \mathscr{X}^*: | \langle a, x^*\rangle | \le 1 \text{ for all } a \text{ in } A\}$

To show that $A^{\circ}$ is balanced, given $x^*$ in $A^{\circ}$ and $|\alpha| \le 1$, we have to show that $\alpha x^*$ is also in $A^{\circ}$.

If $|\langle a, x^*\rangle | \le 1$, then $|\langle a, \alpha x^*\rangle | = |\alpha |  | \langle a, x^* \rangle | \le 1$, hence $\alpha x^*$ is in $A^{\circ}$

To show that $A^{\circ}$ is convex, we have to show that given arbitrary $x_1^*$ and $x_2^*$ in $A^{\circ}$, then an arbitrarily chosen $x_3^*$ in $\{t x_1^* + (1-t) x_2^* :0\le t \le 1\}$ is also in $A^{\circ}$, say $t_3 x_1^* + (1-t_3) x_2^*$ (where $t_3$ was chosen arbitrarily in $0\le t \le 1\}$).

Hence we need to prove that, for all $a$ in $A$, $|\langle a, t_3 x_1^* + (1-t_3) x_2^* \rangle | \le 1$

If $|\langle a, x_1^*\rangle | \le 1$ and $|\langle a, x_2^*\rangle | \le 1$, then simply by multiplication of both sides $t_3|\langle a, x_1^*\rangle | \le t_3$ and $(1-t_3)|\langle a, x_2^*\rangle | \le 1-t_3$ since $0 \le t_3 \le 1$ (to not flip the signs). Adding the inequalities we get:

\begin{align}
t_3|\langle a, x_1^*\rangle | +(1-t_3)|\langle a, x_2^*\rangle | \le 1 \\
|\langle a, t_3 x_1^*\rangle | +|\langle a, (1-t_3) x_2^*\rangle | \le 1 \\
|\langle a, t_3 x_1^* + (1-t_3) x_2^*\rangle | \le 1
\end{align}

Both 10) and 11) by linearity. Note that (9) could also have been obtained geometrically, by noting that the unit disk is convex, so any point between any two points would also be in the unit disk.

\subsection{If $A_1 \subseteq A$, then $A^{\circ} \subseteq A^{\circ}_1$}

If $x^* \in A^{\circ}$, then $| \langle a,x^* \rangle | \le 1$ for all $a$ in $A$. Since $A_1 \subseteq A$,  $| \langle a,x^* \rangle | \le 1$ for all $a$ in $A_1$ as well. Therefore $x^* \in A^{\circ}$. Since $x^*$ was chosen arbitrarily,  $A^{\circ} \subseteq A^{\circ}_1$.


\section{If $A\subseteq\mathscr{X}$, show that $A$ is weakly bounded if and only if $A^{\circ}$ is absorbing in $\mathscr{X}^*$}

This works equally well at any point $x$ in $\mathscr{X}$ so assume that $0 \in A$ and  $A^{\circ}$ is absorbing in $\mathscr{X}^*$ at $0$.

Assume $A$ is weakly bounded. Thus for every $x^*$ in $\mathscr{X}$, $x^*(A)$ is bounded in $\mathbb{C}$. Take an arbitrary $x^*_0$ such that $x^*_0(A)$ is bounded by $M_0$. Take $\varepsilon_0 = 1 / M_0$. We must show that for $0 \le t < \varepsilon_0$, $tx^* \in A^{\circ}$. Indeed, for all $a$ in $A$, 

\begin{align}
|\langle a, x^* \rangle | \le M_0 \\
|\langle a, t x^* \rangle | \le \varepsilon_0 
\end{align}

13) by linearity. Thus all of $t x^*$ is in $A^{\circ}$, so $A^{\circ}$ is absorbing.


Assume $A^{\circ}$ is absorbing in $\mathscr{X}$. Thus for each $x^*$ in $\mathscr{X}$ there is an $\varepsilon > 0$ such that $tx^* \in A^{\circ}$ for $0 \le t < \varepsilon$, i.e. $| \langle a, t x^* \rangle | \le 1$. Thus, by linearity, $x^*$ cannot be larger than $1/\varepsilon$, and is thus bounded.

\end{document}