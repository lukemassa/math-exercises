\documentclass{article}

\usepackage{mathrsfs,amssymb,amsmath}

\begin{document}

\section{Show that the particular point topology is $T_0$ but not $T_1$}

Consider the particular point topology of a set $X$ cardinality $\ge 3$ with particular point $x$, that is all sets containing $x$ are open as is the empty set, but no other sets.

Consider two arbitrary "other" points $x_1$ and $x_2$. 

To show $X$ is $T_0$ consider the two cases:

Case 1: $x$ and $x_1$. $x$ has an open neighborhood $\{x\}$ that does not contain $x_1$.

Case 2: $x_1$ and $x_2$. $x_1$ has an open neighborhood $\{x, x_1\}$ that does not contain $x_2$.

This is sufficient to prove the statement for all pairs of points, since $x_1$ and $x_2$ were chosen arbitrarily, and every pair of points falls into one of these cases.

To show $X$ is not $T_1$, consider the case $x$ and $x_1$. Every open set containing $x_1$ will also contain $x$ (since all non-empty open sets contain $x$)

\section{Show that the particular point topology is not Hausdorf}

Obviously if a space is not $T_1$ it can't be $T_2$.

A more illustrative proof starts with the fact limits in Hausdorf are unique. So we must construct a sequence that converges to more than one point.

Consider any sequence $s_n$ that is eventually $x$. This sequence has $x$ as a limit, but it also has every other point in the space. In the case of $x_1$, every neighborhood contains of $\{x, x_1\}$, which $s_n$ eventually never leaves.

\end{document}
