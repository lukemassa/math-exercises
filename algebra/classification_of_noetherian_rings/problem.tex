\documentclass{article}

\usepackage{mathrsfs,amssymb,yfonts}

\begin{document}

\section{Show the relationship between Commutative Noetherian Rings (CNDs) and the class inclusions on wikipedia}

\begin{center}
\begin{tabular}{ |c|c| }
 \hline
 Class & Relationship to Noetherian Rings\\ 
 \hline
 Commutative Rings & $\supset$ \\ 
  Integral Domains & Incomparable  \\ 
  Integrally closed domains & Incomparable  \\ 
  GCD domains & Incomparable  \\ 
  Unique Factorization domains & Incomparable  \\ 
  Principle Ideal Domains & $\subset$  \\ 
  Euclidean Domains & $\subset$  \\ 
  Fields & $\subset$  \\ 
 \hline
\end{tabular}
\end{center}

\subsection{Let $(X_0, X_2, ..., X_{n+1})$ be a finite sequence of sets such that $X_{i+1} \subset X_i$ for all $i$ where $0 \le i \le n$ and a set $Y$ such that $X_0 \supset Y$ and $X_{n+1} \subset Y$. Then there exists integers $j, k, l \ge 0$ such that $j+k+l=n$ and the first $j$ sets of X contain Y, the next $k$ are incomparable to Y in the sense that they intersect but neither is the subset of the other, and the last $l$ are contained by Y}

Two non-identical sets X and Y are in one of four relations: subset, superset, incomparable, and disjoint. If $X_i$ is a subset or disjoint of $Y$, then $X_{i+1}$ must be a subset or disjoint respectively. If $X_i$ is a superset of $Y$, then $X_{i+1}$ could be any four options. And if $X_i$ is incomparable with $Y$, then all but superset are options.

Putting this together, we start with a superset followed by zero or more supersets, then we then transition to zero or more incomparables. Because we know we end with a subset, there can be no disjoints, since once $X_i$ is disjoint all sets after it must be disjoint.

\subsection{1 follows the setup of 1.1}

$X_0$ is Commutative Rings, $Y$ is Noetherian Rings, $n=6$, and $X_7$ is Fields. Clearly a Commutative Noetherian Ring is a commutative ring, so $X_0 \subset Y$. A field is a commutative Noetherian Ring since in an Noetherian Ring every ideal is finitely generated, and in a field the only two ideals are (0) and (1), both generated by a single element.

\subsection{In the language of 1.1, $j=0$, $k=4$, and $l=2$}

To show this, we just need to show that Integral Domains and UFDs are Incomparable, and that PIDs are subsets, since those are the transitions.

\subsection{Integral Domains and CNDs are Incomparable}

The zero ring is Noetherian, since its only ideal is (0). However it is not an Integral Domain, as it is explicitly excluded in the definition.

The ring of the integers if Noetherian since each of its ideals are generated by a single element, and it isn Integral Domain since any two non-zero elements multiplied together is non-zero.

The polynomial ring over countably infinite unknowns is an integral domain, since any two non-zero polynomials of degree $n$ and $m$ multiplied together will have degree $n+m$, and therefore not be zero. However, it is not Noetherian, since the chain of strictly inclusive ideals $(X_1), (X_1, X_2), (X_1, X_2, X_3), ... $ does not terminate.

\subsection{Unique Factorization Domains and CNDs are Incomparable}

UFDs and CNDs are not disjoint and CNDs are not a subset of UFDs for the same reason above. What remains to show is that there exists a UFD that is not Noetherian.

The ring $Z[\sqrt -5]$ is not a UFD since 6 can be written as $2 \times 3$ or $(1+ \sqrt -5)(1- \sqrt -5)$. However $Z[\sqrt -5]$ is Noetherian since its Krull dimension is 2 (how to prove this?)


\subsection{Principle Ideal Domains are CNDs}

PIDs are commutative, and each ideal is generated by a single element, hence finitely generated.

\end{document}