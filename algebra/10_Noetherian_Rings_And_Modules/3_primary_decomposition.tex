\documentclass{article}

\usepackage{mathrsfs,amssymb,yfonts}

\begin{document}

\section{Let $A$ be a commutative ring. Let $M$ be a module, and $N$ a submodule. Let $N=Q_1 \cap ... \cap Q_r$ be a primary decomposition of $N$. Let $\bar{Q}_i = Q_i/N$. Show that $0 = \bar{Q}_1 \cap ... \bar{Q}_r$ is a primary decomposition of $0$ in $M/N$. State and prove the converse.}

1. Each $\bar{Q_i}$  is primary

Given $a \in A$, $a_{M/Q_i}$ is either injective or nilpotent, we must show that, given $a \in A$, $a_{ (M/N) / ( Q_i / N)}$ is either injective or nilpotent.

The function $a_{M/Q_i}$ is a particular function from $M/Q_i$ to itself. Via the isomorphism $\sigma: M/Q_i \mapsto (M/N) / ( Q_i / N)$, define $\hat{a}$ as a function from  $(M/N) / ( Q_i / N)$ to itself (Is just quoting the isomorphism theorem sufficient here?)

Since they are both multiplication by $a$, $a_{ (M/N) / ( Q_i / N)}$ and $\hat{a}$ are the same function on  $(M/N) / ( Q_i / N)$. Thus if $a_{M/Q_i}$ is injective (resp. nilpotent) then  $a_{ (M/N) / ( Q_i / N)}$ is injective (resp. nilpotent).

\noindent
2. Their intersection is $0 = \bar{Q}_1 \cap ... \cap \bar{Q}_r$

Assume this is false.

Take element $a \ne (0) \in \bar{Q}_1 \cap ... \cap \bar{Q}_r$. (Better way to write this?) The pre-image of $a$ under the canonical homomorphism $M \mapsto M/N$ is also not in $N$, since $N$ is exactly the kernel of this homomorphism. 

However it $a$ *is* in each $\bar{Q}_i$ so its preimage has to be in $Q_i$, so has to be in $N$, a contradiction.


\noindent
3. If 0 is primary decom, then N is primary decomp

Showing that $Q_i$ is primary given that $\bar{Q}_i$ is primary is identical to part 1 by following the isomorphism $ M / Q_i \cong (M/N)/(Q_i/N)$ the other direction.

The proof that the intersection is N is also directly analogous to part 2, shown here:

Assume it is false.

Take some $x$ not in $N$ but in $N=Q_1 \cap ... \cap Q_r$. Under the canonical homomorphism the image of $x$ is in each of $\bar{Q}$, thus is $0$. However this is a contradiction since the image of $x$ is non-zero.


\section{Let $\mathfrak{p}$ be a prime ideal and $\mathfrak{a}$, $\mathfrak{b}$ be ideals of $A$. If $\mathfrak{a b} \subset \mathfrak{p}$, show that $\mathfrak{a} \subset \mathfrak{p}$ or $\mathfrak{b} \subset \mathfrak{p}.$}

If $\mathfrak{a} \not\subset \mathfrak{p}$ and $\mathfrak{b} \not\subset \mathfrak{p}$, prove that $\mathfrak{ab} \not\subset \mathfrak{p}$

Pick $a$ in $\mathfrak{a}$ and $b$ in $\mathfrak{b}$ but not in $\mathfrak{p}$. Since $\mathfrak{p}$ is prime, $ab$ cannot be in $\mathfrak{p}$.


\section{Let $\mathfrak{q}$ be a primary ideal. Let $\mathfrak{a}$, $\mathfrak{b}$ be ideals, and assume $\mathfrak{ab} \subset \mathfrak{q}$. Assume that $\mathfrak{b}$ is finitely generated. Show that $\mathfrak{a} \subset  \mathfrak{q}$ or there exists some positive integer $n$ such that $\mathfrak{b}^n \subset \mathfrak{q}$.}

Assume that $\mathfrak{a} \not\subset \mathfrak{q}$. We will show that this implies there there exists some positive integer $n$ such that $\mathfrak{b}^n \subset \mathfrak{q}$. Take $a_0$ to be in $\mathfrak{a}$ but not in $\mathfrak{q}$. 

An arbitrary element of $\mathfrak{b}$, being finitely generated, looks like $k_1 b_1 + k_2 b_2 + ... + k_r b_r$ (where $k_i$ are positive integers and $r$ and $n$ are unrelated). For each $k_i b_i$, since $a_0 (k_i b_i)$ in $\mathfrak{q}$ but $a_0$ is not and $\mathfrak{q}$ is primary, $(k_i b_i)^{n_i} \in \mathfrak{q}$ for some finite $n_i$. Our goal is to find some $n$ large enough such that every term of $(k_1 b_1 + k_2 b_2 + ... + k_r b_r)^n$ is in $\mathfrak{q}$, since, being additively closed, that would imply $b^n$ itself were in $\mathfrak{q}$.

Take $n=\prod_{i=1}^{r} n_i$. Each term of $(k_1 b_1 + k_2 b_2 + ... + k_r b_r)^n$ is a homogenous monomial of degree $n$, thus in general looks like $\prod_{i=1}^{r} (k_i b_i)^{m_i}$ where the $m_i$'s sum to $n$.

If any $m_i \ge n_i$, then the term itself will be in $\mathfrak{q}$, since $(k_i b_i)^{n_i}$ is in $\mathfrak{q}$ as are its subsequent powers and anything multiplied by it, since ideals absorb multiplication. 

It is impossible for all $m_i < n_i$, since then they would not fully sum to $n$, so therefore at least one is and the term is in $\mathfrak{q}$. Since $r$ is finite, so too is $n$, thus proving the statement.

\section{Let $A$ be Noetherian and let $\mathfrak{q}$ be a $\mathfrak{p}-$primary ideal. Show that there exists some $n \ge 1$ such that $\mathfrak{p}^n \subset \mathfrak{q}$}

$\mathfrak{q}$ being $\mathfrak{p}$-primary means that the radical of $\mathfrak{q}$ is $\mathfrak{p}$. Hence $\sqrt{\mathfrak{q}} = \{r \in A | r^n \in \mathfrak{q}\} = \mathfrak{p}$

Since $A$ is Noetherian, $\mathfrak{p}$ is finitely generated, so an arbitrarily element of $p$ is of the form $k_1 p1 + k_2 p_2 + ... + k_r p_r$, where $k_i$ and $r$ are positive integers. 
 
For every $k_i p_i$, there's an $n_i$ such that $(k_i p_i)^{n_i}$ is in $\mathfrak{q}$. For the same reasoning as 3) above, $n=\prod_{i=1}^{r} n_i$ is such that this arbitrary $p$ is in $\mathfrak{q}$.

\break
\break

8. Let $A$ be a local ring. Show that any idempotent $\ne 0$ in $A$ is necessary the unit element. 

We have to show that $e * e = e \Rightarrow e = 1$

If $e$ is a unit, we're done (Left multiplication by $e^{-1}$ shows $e = 1$).

So assume $e$ is not a unit. Then it must be in the maximal ideal $\mathfrak{m}$ (otherwise it would generate a proper ideal not contained in $\mathfrak{m}$). Not quite sure how to finish...

\end{document}